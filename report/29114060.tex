\documentclass[uplatex,12pt]{jsarticle}
\usepackage[dvipdfmx]{graphicx}
\usepackage{url}
\usepackage{listings,jlisting}
\usepackage{ascmac}
\usepackage{amsmath,amssymb}

%ここからソースコードの表示に関する設定
\lstset{
%  basicstyle={\ttfamily},
  basicstyle={\small},
  identifierstyle={\small},
%  commentstyle={\smallitshape},
%  commentstyle={\small\itshape},
  commentstyle={\small\ttfamily},
  keywordstyle={\small\bfseries},
  ndkeywordstyle={\small},
  stringstyle={\small\ttfamily},
  frame={tb},
  breaklines=true,
  columns=[l]{fullflexible},
  numbers=left,
  xrightmargin=0zw,
  xleftmargin=3zw,
  numberstyle={\scriptsize},
  stepnumber=1,
  numbersep=1zw,
  lineskip=-0.5ex
}
%ここまでソースコードの表示に関する設定

\title{知能プログラミング演習II 課題2}
\author{グループ8\\
  29114060 後藤 拓也\\
}
\date{2019年10月28日}

\begin{document}
\maketitle

\paragraph{提出物} rep3

\paragraph{グループ} グループ8

\paragraph{グループメンバー}
\begin{center}
\begin{tabular}{|c|c|c|}
  \hline
  学生番号&氏名&貢献度比率\\
  \hline\hline
  29114003&青山周平&no\\
  \hline
  29114060&後藤拓也&no\\
  \hline
  29114116&増田大輝&no\\
  \hline
  29114142&湯浅範子&no\\
  \hline
  29119016&小中祐希&no\\
  \hline
\end{tabular}
\end{center}
\paragraph{自分の役割} 必須課題3.3
\\ 「知識システムの質問応答システム」
%%%%%%%%%%%%%%%%%%%%%%%%%%%%%%%%%%%%%%%%%%%%%%%%%%%%%%%%%%%%%%%%%%%%%%%%%%%%%%
\section{課題の説明}
\begin{screen}
課題3-1または3-2で作った知識表現を用いた質問応答システムを作成せよ.
なお,ユーザの質問は英語や日本語のような自然言語が望ましいが,難しければ課題2で扱ったような変数を含むパターン (クエリー) でも構わない.
\end{screen}

\section{手法}

\begin{enumerate}
\item 課題2で扱ったおうな変数を含むクエリーによる質問
\item 英語による質問
\end{enumerate}


1.に関して, 

2.に関しては,


\section{実装}

手法1に関する「課題2で扱ったおうな変数を含むクエリーによる質問」の部分を実装したソースコード\ref{src:No1}に示す。
\begin{lstlisting}[caption=1文すべて終わったら格納する,label=src:No1]
/***
 * 課題2で扱ったような変数を含むパターン (クエリー)による質問応答システム
 * "?x is-a sports"と"?y hobby ?x"をとらえる
 * → 質問は3つのトークンに分けられる
 */
Scanner stdIn1 = new Scanner(System.in);	//文字列読み込み
Scanner stdIn2 = new Scanner(System.in);	//数値読み込み
ArrayList<ArrayList<String>> queryList = new ArrayList<ArrayList<String>>(); //質問(query)を入れる
StringTokenizer st;		//トークンごとに分解
int retry;
do {
	ArrayList<String> tokenList = new ArrayList<>();
	System.out.println("質問を入力してください");
	String s = stdIn1.nextLine(); 	//質問文がここに入り,
	st = new StringTokenizer(s);	//トークンごとに分解し,
	for(int i=0; i<st.countTokens(); i++) {
		tokenList.add(st.nextToken());
	}
	tokenList.add(st.nextToken());
	queryList.add(tokenList);
	System.out.println("もう1つ? 1...Yes/ 0...No");
	retry = stdIn2.nextInt();
}while(retry == 1);

ArrayList<Link> query = new ArrayList<Link>();
for(int i=0; i<queryList.size(); i++) {
	query.add(new Link(queryList.get(i).get(1), queryList.get(i).get(0), queryList.get(i).get(2)));
}
sn.query(query);
\end{lstlisting}

flagはフィールド変数として, Unifierクラスのどのメソッドでも用いられる. 上記には書かれていないが, tokenMatchingメソッドのvarMatchingメソッド内で1単語(トークン)をマッチングさせるたびに, マッチング成功の成否に合わせflag管理をしている.

また, ハッシュマップは実際には2次元リストにより構築されているため, 1文解析がすべて終わり, フラグが立ってない場合の処理だが, 手法5の説明時には簡略化のために省略したが, 実際には"ハッシュマップに格納する"のではなく, "ハッシュマップに格納するためのリストvarslistに登録する"である.\\\\


\section{実行例}
日本語で言うと, [スポーツを趣味にしている人はだれか?]と質問したときの実行結果が以下のようになる.
\begin{lstlisting}
Successfully started
検索結果を取得
質問を入力してください
?x is-a sports
もう1つ? 1...Yes/ 0...No
1
質問を入力してください
?y hobby ?x
もう1つ? 1...Yes/ 0...No
0
*** Query ***
?x  =is-a=>  sports
?y  =hobby=>  ?x
[{?x=baseball, ?y=Taro}]
\end{lstlisting}

まずはスポーツが何かを求め, その後, それを趣味としている人を探す.

正しい関係性が出力されていることが確認される.\\

\section{考察}
手法1のやり方では, 1つの質問しかできない. というのも, 2つ入力しているが, これは"かつ"の関係で結ばれている1つの質問文なのである.

違和感を覚えたのは, 「Taro hobby baseball」という例文. 何となく「太郎の趣味は野球です」になるが, Google翻訳にかけたら, 「ヒロキホビーサッカー」である. そもそもhobbyは動詞になり得ない. 正しくは, 「Taro ’s hobby is baseball」である. この1文を扱うのは相当難しい. ただ, 「Taro hobby baseball」が単純に, 「Tail, Label, Head」を表していると考えれば, 何とかなるかもしれない.

\section{感想}
Javaの使い方は, ググってもいいが, 昔しっかり使い込んだ教科書に立ち戻るのも, また一挙である.

%%%%%%%%%%%%%%%%%%%%%%%%%%%%%%%%%%%%%%%%%%%%%%%%%%%%%%%%%%%%%%%%%%%%%%%%%%%%%%
% 参考文献
\begin{thebibliography}{99}
\bibitem{notty} Javaによる知能プログラミング入門 --著:新谷 虎松 \\
\bibitem{notty} 新・明解 Java 入門 --著:柴田望洋 \\
\bibitem{notty} Java 指定型の読み取り --著:Let's プログラミング \\
\url{https://www.javadrive.jp/start/scanner/index2.html}
\bibitem{notty} Google翻訳 --著:Google \\
\url{https://translate.google.com/?hl=ja}
\end{thebibliography}

\end{document}